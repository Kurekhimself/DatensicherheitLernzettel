\documentclass[a4paper,12pt]{article}
\usepackage[utf8]{inputenc}
\usepackage[ngerman]{babel}
\usepackage{enumitem}
\usepackage{geometry}
\usepackage{multicol}
\usepackage{parskip}
\geometry{margin=2.5cm}

\title{Lernkarten – Datensicherheit}
\author{}
\date{}

\begin{document}

\maketitle

\section*{Begriffsdefinitionen}

\begin{itemize}[leftmargin=*]
  \item \textbf{Frage:} Was bedeutet Vertraulichkeit in der IT-Sicherheit?\\
  \textbf{Antwort:} Schutz vor unzulässiger Kenntnisnahme oder unbefugtem Zugriff.

  \item \textbf{Frage:} Was ist unter Integrität zu verstehen?\\
  \textbf{Antwort:} Schutz vor unzulässigen Veränderungen von Daten.

  \item \textbf{Frage:} Was bedeutet Verfügbarkeit?\\
  \textbf{Antwort:} Informationen und Systeme sind bei Bedarf nutzbar.

  \item \textbf{Frage:} Was ist Authentizität?\\
  \textbf{Antwort:} Die nachprüfbare Echtheit eines Subjekts oder Objekts.

  \item \textbf{Frage:} Was bedeutet Verbindlichkeit?\\
  \textbf{Antwort:} Handlungen können eindeutig zugeordnet werden.

  \item \textbf{Frage:} Was unterscheidet Security und Safety?\\
  \textbf{Antwort:} 
  \begin{itemize}
    \item \textbf{Security:} Schutz vor äußeren Bedrohungen (z. B. durch Angreifer)
    \item \textbf{Safety:} Schutz vor Gefahren, die vom System selbst ausgehen
  \end{itemize}

  \item \textbf{Frage:} Was besagt die Kettenregel der Sicherheit?\\
  \textbf{Antwort:} Eine Sicherheitskette ist nur so stark wie ihr schwächstes Glied.

  \item \textbf{Frage:} Nenne Typen von Angreifern.\\
  \textbf{Antwort:} Innentäter, Hacker, Script-Kiddies, Wirtschaftsspione, Geheimdienste.

  \item \textbf{Frage:} Nenne typische Angriffsarten.\\
  \textbf{Antwort:} Sniffing, Spoofing, Man-in-the-Middle, (D)DoS.
\end{itemize}

\section*{Kryptologie}

\subsection*{Kodierung vs. Chiffrierung}
\begin{itemize}[leftmargin=*]
  \item \textbf{Frage:} Was unterscheidet Kodierung und Chiffrierung?\\
  \textbf{Antwort:} 
  \begin{itemize}
    \item \textbf{Kodierung:} eindeutige Zuordnung \(f(x) = y\)
    \item \textbf{Chiffrierung:} abhängig vom Schlüssel \(f(x, k) = y\)
  \end{itemize}
\end{itemize}

\subsection*{Klassische Chiffren}
\begin{itemize}[leftmargin=*]
  \item \textbf{Frage:} Was ist eine Cäsar-Verschlüsselung?\\
  \textbf{Antwort:} Substitution, bei der Buchstaben um eine feste Anzahl Stellen verschoben werden (z. B. A → D bei +3).

  \item \textbf{Frage:} Welche Chiffren zählen zu Substitutionschiffren?\\
  \textbf{Antwort:} Cäsar, Vigenère, Playfair, Codebuch.

  \item \textbf{Frage:} Welche zählen zu Transpositionschiffren?\\
  \textbf{Antwort:} Skytale, Doppelwürfel.
\end{itemize}

\subsection*{Moderne Chiffren}
\begin{itemize}[leftmargin=*]
  \item \textbf{Frage:} Welche Eigenschaften hat die XOR-Verknüpfung?\\
  \textbf{Antwort:}
  \begin{itemize}
    \item Kommutativ: \(A \oplus B = B \oplus A\)
    \item Assoziativ: \(A \oplus (B \oplus C) = (A \oplus B) \oplus C\)
    \item Neutrales Element: \(A \oplus 0 = A\)
    \item Selbstinvers: \(A \oplus A = 0\)
  \end{itemize}

  \item \textbf{Frage:} Was ist ein One-Time Pad?\\
  \textbf{Antwort:}
  \begin{itemize}
    \item Zufälliger Schlüssel, genauso lang wie die Nachricht
    \item Nur dem Sender/Empfänger bekannt
    \item Wird nie wiederverwendet
    \item Theoretisch unknackbar
  \end{itemize}
\end{itemize}

\subsection*{Strom- vs. Blockchiffren}
\begin{itemize}[leftmargin=*]
  \item \textbf{Frage:} Was unterscheidet Strom- und Blockchiffren?\\
  \textbf{Antwort:}
  \begin{itemize}
    \item \textbf{Stromchiffren:} Bitweise Verschlüsselung (z. B. ChaCha20)
    \item \textbf{Blockchiffren:} Verschlüsselung in Blöcken (z. B. AES, DES)
  \end{itemize}
\end{itemize}

\subsection*{AES vs. DES}
\begin{itemize}[leftmargin=*]
  \item \textbf{Frage:} Nenne drei Unterschiede zwischen AES und DES.\\
  \textbf{Antwort:}
  \begin{itemize}
    \item Schlüssellänge: DES: 56 Bit, AES: 128–256 Bit
    \item Blockgröße: DES: 64 Bit, AES: 128 Bit
    \item Sicherheit: DES veraltet, AES gilt als sicher
  \end{itemize}
\end{itemize}

\subsection*{Blockchiffren-Betriebsmodi}
\begin{itemize}[leftmargin=*]
  \item \textbf{Frage:} Was ist ein Nachteil des ECB-Modus?\\
  \textbf{Antwort:} Gleiche Klartextblöcke ergeben gleiche Chiffretextblöcke – kein Schutz vor Mustererkennung.

  \item \textbf{Frage:} Wie funktioniert der CBC-Modus?\\
  \textbf{Antwort:} Jeder Block wird mit dem vorherigen Chiffretextblock per XOR verknüpft; für den ersten Block wird ein Initialisierungsvektor (IV) verwendet.
\end{itemize}

\section*{Asymmetrische Kryptographie – RSA}

\begin{itemize}[leftmargin=*]
  \item \textbf{Frage:} Welche mathematischen Konzepte liegen RSA zugrunde?\\
  \textbf{Antwort:} 
  \begin{itemize}
    \item Modulares Rechnen
    \item Multiplikative Inverse
    \item Modulares Potenzieren
    \item Satz von Euler
  \end{itemize}

  \item \textbf{Frage:} Was bedeutet \(e \cdot d \equiv 1 \mod \varphi(n)\)?\\
  \textbf{Antwort:} \(e\) und \(d\) sind zueinander multiplikativ inverse modulo \(\varphi(n)\); Voraussetzung für RSA.

  \item \textbf{Frage:} Was ist der Unterschied zwischen öffentlichem und privatem Schlüssel bei RSA?\\
  \textbf{Antwort:}
  \begin{itemize}
    \item \textbf{Öffentlicher Schlüssel:} \((n, e)\), bekannt für alle
    \item \textbf{Privater Schlüssel:} \((n, d)\), geheim
  \end{itemize}
\end{itemize}

\end{document}
